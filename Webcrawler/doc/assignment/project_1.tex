\documentclass[11pt]{article}
\usepackage[margin=1in]{geometry}
\usepackage{fancyhdr}

\newfont{\sfs}{cmssq8 at 9pt}
\newcommand{\lc}[1]{{\sfs #1}}
\newcommand{\eg}{{e.g.,\ }}
\newcommand{\ie}{{i.e.,\ }}

\lhead{CpSc 215: Software Development Foundations}
\chead{\empty}
\rhead{Spring 2014}
\lfoot{\empty}
\cfoot{\thepage}
\rfoot{\empty}

\setlength{\parindent}{0in}
\setlength{\parskip}{6pt}

\begin{document}
\title{\vspace{-.35in}Project \#1:\\Web Crawler\footnote{This project 
is based on material prepared by Dr. Jason O. Hallstrom}}
\date{\empty}
\maketitle

\pagestyle{fancy}
\thispagestyle{fancy}

\vspace{-.75in}
\section{Objectives}
This project is designed to reinforce a number of learning objectives. 
First, and most important, you will gain experience designing and 
implementing a small (but interesting) Java application. This will reinforce 
your object-oriented design skills, as well as your ability to effectively
apply the Java development tools. More specifically, the project will reinforce 
your understanding of Java interfaces, the Singleton pattern, Java
exceptions, and the Java IO library. It will also improve your ability to
debug Java-based systems.

Before describing the project requirements, two words of caution are 
in order. First, you should be prepared to experience some frustration
with the level of detail provided in the assignment description. In order
to learn object-oriented design skills, you need to be given a chance 
to do some design --- which means that all of the implementation steps 
can't be laid out for you in advance. This level of flexibility is more in line
with what you will encounter when you are employed as a developer.

The second warning is that this is a time-consuming assignment. You 
are expected to take the time to investigate the Java class libraries. 
Some of the most relevant classes are identified in this document. 
Others could significantly simplify your implementation --- but you're
expected to identify those through independent exploration. You will
also be expected to learn basic HTML syntax, if you haven't done so
already. There are a number of websites devoted to this topic. Have fun!
Explore!

\section{Overview}
For this project, you will implement a simple command-line {\em web crawler}. The
application will accept three arguments. The first will specify the web address from 
which the crawl should begin. The second will specify the maximum crawl depth. Finally, 
the third argument will specify the local directory used to store results from the crawl.

When the crawl begins, it will perform a basic parse of the web page specified
at the command-line. The goal of the parse is to identify the ({\em i}) images 
used within the page, the ({\em ii}) other files linked from within the page\footnote{Excluding 	
web pages and image files}, and the ({\em iii}) web pages linked from
within the page. All three element sets will be saved to disk in the
directory specified at the command-line. The third set of elements will be used
as input to the next step in the page crawl; this process will be applied
recursively up to the maximum specified crawl depth.

It is worth emphasizing that the term {\em ``parse''} is being used
loosely here. You are not required to do any sophisticated HTML 
processing; use the simplest solution that works. The Java String 
library provides more than adequate functionality for completing this
task.  You may also use Regular Expressions or the JSoup Library.

\section{Requirements}
Again, much of the design work is left to you. So while you must satisfy
the requirements listed here, many of the details are left to you. This
is the fun part of programming!

\begin{itemize}\itemsep 0.0in
\item You must follow good design practices throughout your implementation. 
In particular: ({\em i}) Program to interfaces wherever possible. ({\em ii}) Never 
use \lc{public} fields. ({\em iii}) Follow Java naming conventions. ({\em iv})
Simplify your code by eliminating redundancy. ({\em v}) Catch and handle all
checked exceptions.

\item Your main application class must be named \lc{WebCrawler}.

\item Your application must include a \lc{WebPage} class used to represent a
web page. The class must include the following methods: \lc{crawl()}, \lc{getImages()}, 
\lc{getFiles()}, and \lc{getWebPages()}. The \lc{crawl()} method is used to parse the 
web page represented by the target object. The remaining methods are used
to retrieve the results of the parse. Hence, the \lc{get} methods must only 
be invoked after \lc{crawl()} has terminated.

\item Your application must include classes for representing the images, files,
and web pages linked from within a page. These classes must be named
\lc{WebImage}, \lc{WebFile}, and \lc{WebPage}, respectively. (The last class is
discussed above.) Instances of these classes will be returned by the
\lc{WebPage} {\em getter} methods discussed above.

\item \lc{WebImage}, \lc{WebFile}, and \lc{WebPage} must each implement the
\lc{WebElement} interface. This interface must provide methods to enable a
client component (\ie a user of a \lc{WebElement} instance) to save the
corresponding element to disk. 

\item Your application must implement a {\em singleton} class,
\lc{DownloadRepository}. An instance of this class must be suitably initialized
at system startup using an appropriate \lc{static} initializer. This singleton
must expose an interface that can be used by the crawler class (\lc{WebPage})
to save a \lc{WebElement} object to the repository.
\end{itemize}

%An optional design review will be held the week of February 9$^{th}$. If you choose
%to attend this review, you will only benefit if you've already made some progress on
%your solution.

\section{Submission Instructions}
This project is due before class on Feburary 28th$^{th}$. Absolutely no late assignments
will be accepted.

When your project is complete, archive your solution and use the handin command to submit your
work.  Please submit for your Eclipse project as a .zip archive usin the http://handin.cs.clemson.edu website.
{\bf You must also turn in a hard copy of all of your source materials in class.}

\section{Grading}
Your project will be graded based on the quality of your design and
implementation, as well as the functionality of the application itself. You are
expected to correctly handle all basic web pages, under the assumption that the
web page content is correct. If you encounter an error during processing
---due to improper HTML formatting, embedded scripts that are not handled by
your parser, etc.--- your application must handle the error gracefully --- this
means terminating only the affected branch in your crawl.

This is an intermediate course in software development. Your source materials
should be properly documented. Your source must compile. Your application must
not crash. A violation of any of these requirements will result in an automatic
zero. {\bf Test your application thoroughly.}

\section{Collaboration}
You may work independently or with one partner. You must {\em not} discuss the
problem or the solution with classmates outside of your group. Keep this in
mind before you choose to work independently. 

{\bf Collaborating in any manner with peers outside of your group will be treated as
academic misconduct.}
\end{document}
